\documentclass[12pt]{paper}

\usepackage[a4paper, left=20mm, right=20mm, top=20mm, bottom=20mm]{geometry}
\usepackage[bottom]{footmisc}
\usepackage{amsmath}
\usepackage{amsfonts}
\usepackage{kotex}
\usepackage{amsthm}
\usepackage{amssymb}
\usepackage{mathtools}
\usepackage{setspace}
\usepackage{graphicx}


\title{현대대수1및실습}

\author{임기정}

\newenvironment{context}[1][]{\noindent \textbf{{#1}.}}{\hfill $ \dashv $}

\begin{document}

  \nocite{fraleigh2009}

  \setstretch{1.5}
  \maketitle
  \hspace{12pt}
  
  임복희 교수님께서 담당하시는 2021년 1학기 현대대수1및실습 1 분반 안의, Noether 조의 이름으로 작성된 노트 정리입니다.
  딱딱한 현대대수학을 최대한 쉽게 풀어 쓰고자 노력하였습니다.

  \section{1주차 노트 정리}
  \hspace{12pt}

  현대대수1및실습을 수강하셨다면 마땅히 알아야 할 ``군''이라는 낱말의 뜻에 대하여 논하기 위해서는, \underline{이항 구조}(\textit{binary structure})부터 알아야 합니다.

  집합 $S$가 이항 구조가 되기 위해서는 \underline{이항 연산}(\textit{binary operator}) $*$을 가져야 합니다.
  이때 $*$이 이항 연산이라 함\footnote{참고 문헌 \cite{fraleigh2009} 20쪽, 정의 2.1 인용}은 그것이
  \begin{enumerate}
    \item[(a)] $S \times S$를 \underline{정의역}(\textit{domain})으로 가지면서
    \item[(b)] $S$를 \underline{공역}(\textit{codomain})를 가져야 하며
    \item[(c)] \underline{잘 정의된}(\textit{well defined}) 
  \end{enumerate}
  함수이어야 함을 의미합니다.
  각 조건에 대하여 설명하자면 다음과 같습니다:
  \begin{enumerate}
    \item 조건 (a)의 뜻은 아무 $a \in S$와 아무 $b \in S$에 대하여 순서쌍 $\left( a , b \right)$를 $*$에게 입력으로 줄 수 있어야 한다는 것이고;
    \item 조건 (b)의 뜻은 입력 $\left( a , b \right)$에 대한 $*$의 출력 $a * b$이 항상 $S$의 원소이어야 한다는 것이며;
    \item 조건 (c)의 뜻은 그 출력이 딱 하나로 결정되어야 한다는 것입니다.
  \end{enumerate}
  
  이때 $*$이 조건 (a)를 어긴다면 $*$가 $S$ 위의 \underline{모든 것들에서 정의되지 않는다}(\textit{not everywhere defined})고 하고,
  조건 (b)를 어긴다면 $S$는 \underline{$*$ 아래에서 닫혀 있지 않다}(\textit{not closed under $*$})고 하며,
  조건 (c)를 어긴다면 $*$이 \underline{잘 정의되지 않았다}(\textit{not well defined})고 합니다.

  만약 $*$이 이 모든 조건들을 지킨다면 이항 연산이 되는데,
  이때 이항 연산이 된다는 사실을
  \begin{align*}
    S \times S & \xrightarrow{*} S \\
    \left( a , b \right) & \mapsto a * b
  \end{align*}
  와 같이 적을 수 있습니다.

  이때 집합 $S$가 이항 연산 $* : S \times S \to S$을 가지면 순서쌍 $\left( S , * \right)$을 이항 구조라고 부릅니다.
  참고로 ``이항 구조''의 동의어에는 ``\underline{아군}''(\textit{groupoid})와 ``\underline{마그마}''(\textit{magma})가 있습니다.\footnote{참고 문헌 \cite{fraleigh2009} 38쪽 인용}

  마그마 $ \left( S , * \right) $가 \underline{군}\footnote{참고 문헌 \cite{fraleigh2009} 37쪽, 정의 4.1 인용}(\textit{group})이 되기 위해서는 다음 세 조건을 만족시켜야 합니다:
  \begin{enumerate}
    \item \underline{결합 법칙}(\textit{associative law})\footnote{참고 문헌 \cite{fraleigh2009} 23쪽, 정의 2.12 인용}이 성립해야 함:
    \begin{equation*}
      \left( \forall a \in S \right) \left( \forall b \in S \right) \left( \forall c \in S \right) \left[ \left( a * b \right) * c = a * \left( b * c \right) \right] \tag{1}
    \end{equation*}
    이어야 합니다.
    다시 말해, 모든 $a , b , c \in S$에 대하여 $\left( a * b \right) * c = a * \left( b * c \right)$이어야 합니다.
    (1)이 성립할 때 그리고 그럴 때에만 ``$*$이 $S$ 위에서 결합적이다''라고 말합니다.

    \item \underline{항등원}(\textit{identity})\footnote{참고 문헌 \cite{fraleigh2009} 32쪽, 정의 3.12 인용}이 존재해야 함:
    적당한 $e \in S$가 존재해서
    \begin{equation*}
      \left( \forall a \in S \right) \left[ e * a = a * e = a \right] \tag{2}
    \end{equation*}
    이어야 합니다.
    이때 $e$를 \underline{$*$에 대한 항등원}(\textit{identity element for $*$})이라고 합니다.
    
    한 가지 주목할 만한 점은 항등원은 많아야 한 개 밖에 존재하지 않는다는 사실입니다.\footnote{참고 문헌 \cite{fraleigh2009} 32쪽, 정리 3.13 인용}
    이 사실을 증명하기 위해서는 $e_1 \in S$가 
    \begin{equation*}
      \left( \forall a \in S \right) \left[ e_1 * a = a * e_1 = a \right] \tag{2-1}
    \end{equation*}
    를 만족한다고 가정한 상태에서 $e = e_1$을 이끌어내면 충분합니다.
    
    이렇게만 해도 충분한 이유를 생각해 봅시다.
    $e_1 \in S$이 $e$와 다르다면 항등원이어서는 안 된다는 걸 보이면,
    자동적으로 항등원은 $e$ 밖에 없게 됩니다.
    그런데 ``$e_1 \in S$이 $e$와 다르다면 항등원이서는 안 된다''의 대우 명제가 바로 ``$e_1 \in S$가 (2-1)을 만족시킨다면 $e = e_1$이다''이기 때문입니다.
    이제 본격적으로 증명을 해보도록 하겠습니다.

    \begin{proof}
      먼저, $e_1 \in S$를 하나 잡은 뒤 그 $e_1$이 (2-1)을 만족시킨다고 가정하겠습니다.
      그러면 (2)의 $a$ 자리에 $e_1$을 넣음으로써 $$e * e_1 = e_1 * e = e_1$$을 얻을 수 있고,
      (2-1)의 $a$ 자리에 $e$를 넣음으로써 $$e_1 * e = e * e_1 = e$$을 얻을 수 있습니다.
      따라서 $e_1 * e = e_1$이고 $e_1 * e = e$임을 알 수 있습니다.
      이것들로부터 $$e = e_1 * e = e_1$$를 얻게 되는데,
      이로써 ($e = e_1$을 보였기 때문에) 증명을 마칠 수 있습니다.
    \end{proof}

    \item 각 원소마다 \underline{역원}(\textit{inverse})\footnote{참고 문헌 \cite{fraleigh2009} 37쪽, 정의 4.1 인용}을 가져야 함:
    각각의 $a \in S$마다, 어떤 $b \in S$가 존재해서
    \begin{equation*}
      a * b = b * a = e \tag{3}
    \end{equation*}
    이어야 합니다.
    이 경우 각각의 $a \in S$에 대하여 (3)을 만족시키는 $b \in S$가 유일하게 존재합니다!
    따라서 그 $b$를 $a^{-1}$과 같이 쓸 수 있고 \underline{$a$의 역원}(\textit{inverse of $a$})라고 부릅니다.

    이를 증명하는 것은 꽤나 어려워 보이지만,
    전과 비슷하게 임의의 역원 후보 $b_1$에 대하여 $b = b_1$일 수 밖에 없음을 보이면 됩니다.

    \begin{proof}
      먼저, $a \in S$를 하나 뽑겠습니다.
      그러면 가정에 의하여 어떤 $b \in S$가 (3)을 만족시킵니다.
      이때 $b_1 \in S$가
      \begin{equation*}
        a * b_1 = b_1 * a = e \tag{3-1}
      \end{equation*}
      를 만족시킨다고 가정한다면,
      \begin{align*}
        b
        & = b * e \tag{by (2)} \\
        & = b * (a * b_1) \tag{by (3-1)} \\
        & = (b * a) * b_1 \tag{by (1)} \\
        & = e * b_1 \tag{by (3)} \\
        & = b_1 \tag{by (2)}
      \end{align*}
      를 얻게 되고,
      증명을 마칠 수 있습니다.
    \end{proof}
  \end{enumerate}

  참고로 결합 법칙을 만족시키는 마그마를 \underline{반군}(\textit{semigroup})이라고 하고,
  항등원이 존재하는 반군을 \underline{모노이드}(\textit{monoid})라고 합니다.\footnote{참고 문헌 \cite{fraleigh2009} 42쪽 인용}
  즉, 군은 각 원소마다 역원을 가지는 모노이드입니다.

  또한 ``\underline{아벨 군}''(\textit{abelian group})의 개념을 소개할까 합니다.
  이항 구조 $\left( S , * \right)$가 주어졌을 때,
  $*$이 \underline{가환}(\textit{commutative})\footnote{참고 문헌 \cite{fraleigh2009} 22쪽, 정의 2.11 인용}하다고 함은
  \begin{equation*}
    \left( \forall a \in S \right) \left( \forall b \in S \right) \left[ a * b = b * a \right] \tag{4}
  \end{equation*}
  이어야 한다는 뜻인데,
  $*$이 가환한 군을 아벨 군이라고 부릅니다.\footnote{참고 문헌 \cite{fraleigh2009} 38쪽, 정의 4.3 인용}

  이제, 군의 예를 하나 공부해 봅시다.

  그 예는 \underline{원군}(\textit{circle group})입니다.
  집합 $S^1$을
  \begin{equation*}
    S^1 := \left\{ z \in \mathbb{C} : \left| z \right| = 1 \right\} \tag{5}
  \end{equation*}
  와 같이 두고,
  복소수의 곱셈으로부터 이항 연산 $* : S^1 \times S^1 \to S^1$을 \underline{유도}(\textit{induce})\footnote{참고 문헌 \cite{fraleigh2009} 21쪽, 정의 2.4 인용}하면,
  $$\left( S^1 , * \right)$$는 군이 됩니다!
  이것이 사실임을 증명할 것인데, 그 전에 다뤄야 할 개념이 있습니다.

  ``$*$은 $S^1$ 위에서 복소수 곱셈의 \underline{유도된 연산}(\textit{induced operation})이다''라고 함은,
  이항 연산 $*$을 복소수의 곱셈의 정의역을 $S^1 \times S^1$으로 줄여서 얻었다는 뜻입니다.
  그런데 이때 $*$이 이항 연산이라는 보장이 있을까요?
  만약 없다면, 우리는 무엇을 확인해 보아야 될까요?

  이항 구조 $\left( S , * \right)$가 주어졌다고 합시다.
  이제 $S$의 부분 집합 $S'$를 하나 잡고, $*'$을
  \begin{align*}
    S' \times S' & \xrightarrow{*'} S' \\
    \left( a , b \right) & \mapsto a * b
  \end{align*}
  로 정의할게요.
  그러면 $*'$이 이항 연산이 되기 위한 조건들 중 (a)와 (c)는, $*$이 이항 연산이라는 가정에 의하여, 이미 성립합니다.
  $*'$는 자신의 정의역에서 $ \left( a , b \right) \mapsto a * b $으로 정의되었으니까요!
  이제 남은 조건은 (b) 뿐인데, 안타깝게도, 이건 성립하지 않을 수도 있습니다.
  따라서 $\left( S' , *' \right)$가 이항 구조임을 보이기 위해서는,
  $S'$에서 (서로 같을 수도 있는) 두 원소 $a$와 $b$를 아무렇게나 뽑아서 $*$에 주더라도,
  그 출력 $a * b$가 $S'$ 밖으로 절대 못 나간다는 것을 보여야 합니다.

  만약 그러하다면, $S'$를 \underline{$*$에 대하여 닫혀 있다}(\textit{closed under $*$})고 합니다.
  즉, 집합 $S' \subseteq S$가 연산 $* : S \times S \to S$에 대하여 닫혀 있을 필요충분조건은
  $$ \left( \forall a \in S' \right) \left( \forall b \in S' \right) \left[ a * b \in S' \right] $$
  가 성립한다는 것입니다.

  이상의 논의로부터 $\left( S^1 , * \right)$가 마그마라는 걸 보이는 것부터 시작해야 되는 걸 알 수 있습니다.
  그렇게 하기 위하여, 두 원소 $z_1 \in S^1$와 $z_2 \in S^1$를 아무렇게나 골라 봅시다.
  그렇더라도, $ z_1 z_2 \in \mathbb{C} $이고 $\left| z_1 \right| = \left| z_2 \right| = 1$이어야 하기 때문에,
  \begin{align*}
    \left| z_1 \right| \left| z_2 \right| = 1
    & \implies \left| z_1 z_2 \right| = 1 \\
    & \implies z_1 z_2 \in S^1
  \end{align*}
  입니다.
  따라서 다음 단계\footnote{$\left( S^1 , * \right)$이 반군임을 증명하기}로 넘어갈 수 있습니다.

  이제 $\left( S^1 , * \right)$이 반군임을 증명해 보겠습니다.
  우리는 이미 복소수의 곱셈이 결합 법칙을 만족시키는 것을 알고 있습니다.
  그런데 $S^1$은 $\mathbb{C}$의 부분 집합입니다.
  그러니까 모든 $a , b , c \in S^1$에 대하여, 
  \begin{equation*}
    (a * b) * c = a * (b * c) \tag{6}
  \end{equation*}
  이 성립하겠죠?
  왜냐하면, (6)을 만족시킨다는 것을 검증받아야 될 모든 삼중쌍 $$ \left( a , b , c \right) \in S^1 \times S^1 \times S^1 $$들은
  전부 집합 $\mathbb{C} \times \mathbb{C} \times \mathbb{C}$의 원소이므로,
  복소수의 곱셈의 결합 법칙이 보증 서주기 때문이죠.
  $*$가 복소수의 곱셈으로부터 유도되었다는 사실을 상기하신다면,
  쉽게 이해가실 거에요.

  이제 $\left( S^1 , * \right)$이 모노이드임을 증명해 보겠습니다.
  $S^1$의 원소들 중 항등원을 하나 찾아서, 그것이 (2)를 만족시킨다는 것을 보이면 충분합니다.
  그런데, 복소수의 곱셈에 대한 항등원인 $1$이 $S^1$에 속해버린 이상, $1$이 $S^1$의 항등원이 되지 않을까요?
  아까와 같은 이유에 의하여, 답은 ``예''입니다.

  드디어 $\left( S^1 , * \right)$이 군임을 증명해 보겠습니다.
  그렇기 위해서는, 각 원소 $a \in S^1$마다 어떤 $b \in S^1$가 (3)을 만족시키는 것을 보이면 충분합니다.
  이제 $a \in S^1$를 하나 잡겠습니다.
  그러면
  \begin{equation}
    \left\{ \begin{aligned}
      a * \left( 1 / a \right) = a \left( 1 / a \right) = 1 \\
      \left( 1 / a \right) * a = \left( 1 / a \right) a = 1
    \end{aligned} \right. \notag
  \end{equation}
  이므로,
  ($1 / a$가 역원임을 보이기 위해서는) $1 / a \in S^1$임만 보이면 충분합니다.
  그런데 $1 / a \in \mathbb{C}$이고 $\left| a \right| = 1$이므로,
  \begin{align*}
    \left| 1 \right| = 1 
    & \implies \left| a \left( 1 / a \right) \right| = 1 \\
    & \implies \left| a \right| \left| 1 / a \right| = 1 \\
    & \implies 1 \left| 1 / a \right| = 1 \\
    & \implies \left| 1 / a \right| = 1 \\
    & \implies 1 / a \in S^1
  \end{align*}
  를 얻게 됩니다.

  이상의 논의로부터 $\left( S^1 , * \right)$를 군이라고 부를 수 있음을 알 수 있습니다.

  \bibliographystyle{unsrt}

  \bibliography{Abstract_Algebra_1_and_Laboratory}

\end{document}
