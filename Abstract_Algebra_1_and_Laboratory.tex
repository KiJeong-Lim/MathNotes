\documentclass[12pt]{paper}

\usepackage[a4paper, left=20mm, right=20mm, top=20mm, bottom=20mm]{geometry}
\usepackage[bottom]{footmisc}
\usepackage{amsmath}
\usepackage{amsfonts}
\usepackage{kotex}
\usepackage{amsthm}
\usepackage{amssymb}
\usepackage{mathtools}
\usepackage{setspace}
\usepackage{graphicx}
\usepackage{tikz}
\usepackage{pgfplots}

\usetikzlibrary{calc,patterns}
\pgfplotsset{compat=newest}

\title{현대대수1및실습}

\author{임기정}

\newenvironment{context}[1][]{\noindent \textbf{{#1}.}}{\hfill $ \dashv $}

\begin{document}

  \nocite{fraleigh2009}

  \setstretch{1.5}
  \maketitle
  \hspace{12pt}
  
  임복희 교수님께서 담당하시는 2021년 1학기 현대대수1및실습 1 분반 안의, Noether 조에서 만든 노트입니다.
  딱딱한 현대대수학을 최대한 쉽게 풀어 쓰고자 노력하였습니다.

  \section{1주차 노트 정리}
  \hspace{12pt}

  현대대수1및실습을 수강하셨다면 마땅히 알아야 할 ``군''이라는 낱말의 뜻에 대하여 논하기 위해서는, \underline{이항 구조}(\textit{binary structure})부터 알아야 합니다.

  집합 $S$가 이항 구조가 되기 위해서는 \underline{이항 연산}(\textit{binary operator}) $*$을 가져야 합니다.
  이때 $*$이 이항 연산이라 함\footnote{참고 문헌 \cite{fraleigh2009} 20쪽, 정의 2.1 인용}은 그것이
  \begin{enumerate}
    \item[(a)] $S \times S$를 \underline{정의역}(\textit{domain})으로 가지면서
    \item[(b)] $S$를 \underline{공역}(\textit{codomain})를 가져야 하며
    \item[(c)] \underline{잘 정의된}(\textit{well defined}) 
  \end{enumerate}
  함수이어야 함을 의미합니다.
  각 조건에 대하여 설명하자면 다음과 같습니다:
  \begin{enumerate}
    \item 조건 (a)의 뜻은 아무 $a \in S$와 아무 $b \in S$에 대하여 순서쌍 $\left( a , b \right)$를 $*$에게 입력으로 줄 수 있어야 한다는 것이고;
    \item 조건 (b)의 뜻은 입력 $\left( a , b \right)$에 대한 $*$의 출력 $a * b$이 항상 $S$의 원소이어야 한다는 것이며;
    \item 조건 (c)의 뜻은 그 출력이 딱 하나로 결정되어야 한다는 것입니다.
  \end{enumerate}
  
  이때 $*$이 조건 (a)를 어긴다면 $*$가 $S$ 위의 \underline{모든 것들에서 정의되지 않는다}(\textit{not everywhere defined})고 하고,
  조건 (b)를 어긴다면 $S$는 \underline{$*$ 아래에서 닫혀 있지 않다}(\textit{not closed under $*$})고 하며,
  조건 (c)를 어긴다면 $*$이 \underline{잘 정의되지 않았다}(\textit{not well defined})고 합니다.

  만약 $*$이 이 모든 조건들을 지킨다면 이항 연산이 되는데,
  이때 이항 연산이 된다는 사실을
  \begin{align*}
    S \times S & \xrightarrow{*} S \\
    \left( a , b \right) & \mapsto a * b
  \end{align*}
  와 같이 적을 수 있습니다.

  이때 공집합이 아닌 집합 $S$가 이항 연산 $* : S \times S \to S$을 가지면 순서쌍 $\left( S , * \right)$을 이항 구조라고 부릅니다.
  참고로 ``이항 구조''의 동의어에는 ``\underline{아군}''(\textit{groupoid})와 ``\underline{마그마}''(\textit{magma})가 있습니다.\footnote{참고 문헌 \cite{fraleigh2009} 38쪽 인용}

  마그마 $ \left( S , * \right) $가 \underline{군}\footnote{참고 문헌 \cite{fraleigh2009} 37쪽, 정의 4.1 인용}(\textit{group})이 되기 위해서는 다음 세 조건을 만족시켜야 합니다:
  \begin{enumerate}
    \item \underline{결합 법칙}(\textit{associative law})\footnote{참고 문헌 \cite{fraleigh2009} 23쪽, 정의 2.12 인용}이 성립해야 함:
    \begin{equation*}
      \left( \forall a \in S \right) \left( \forall b \in S \right) \left( \forall c \in S \right) \left[ \left( a * b \right) * c = a * \left( b * c \right) \right] \tag{1}
    \end{equation*}
    이어야 합니다.
    다시 말해, 모든 $a , b , c \in S$에 대하여 $\left( a * b \right) * c = a * \left( b * c \right)$이어야 합니다.
    (1)이 성립할 때 그리고 그럴 때에만 ``$*$이 $S$ 위에서 결합적이다''라고 말합니다.

    \item \underline{항등원}(\textit{identity})\footnote{참고 문헌 \cite{fraleigh2009} 32쪽, 정의 3.12 인용}이 존재해야 함:
    적당한 $e \in S$가 존재해서
    \begin{equation*}
      \left( \forall a \in S \right) \left[ e * a = a * e = a \right] \tag{2}
    \end{equation*}
    이어야 합니다.
    이때 $e$를 \underline{$*$에 대한 항등원}(\textit{identity element for $*$})이라고 합니다.
    
    한 가지 주목할 만한 점은 항등원은 많아야 한 개 밖에 존재하지 않는다는 사실입니다.\footnote{참고 문헌 \cite{fraleigh2009} 32쪽, 정리 3.13 인용}
    이 사실을 증명하기 위해서는 $e_1 \in S$가 
    \begin{equation*}
      \left( \forall a \in S \right) \left[ e_1 * a = a * e_1 = a \right] \tag{2-1}
    \end{equation*}
    를 만족한다고 가정한 상태에서 $e = e_1$을 이끌어내면 충분합니다.
    
    이렇게만 해도 충분한 이유를 생각해 봅시다.
    $e_1 \in S$이 $e$와 다르다면 항등원이어서는 안 된다는 걸 보이면,
    자동적으로 항등원은 $e$ 밖에 없게 됩니다.
    그런데 ``$e_1 \in S$이 $e$와 다르다면 항등원이서는 안 된다''의 대우 명제가 바로 ``$e_1 \in S$가 (2-1)을 만족시킨다면 $e = e_1$이다''이기 때문입니다.
    이제 본격적으로 증명을 해보도록 하겠습니다.

    \begin{proof}
      먼저, $e_1 \in S$를 하나 잡은 뒤 그 $e_1$이 (2-1)을 만족시킨다고 가정하겠습니다.
      그러면 (2)의 $a$ 자리에 $e_1$을 넣음으로써 $$e * e_1 = e_1 * e = e_1$$을 얻을 수 있고,
      (2-1)의 $a$ 자리에 $e$를 넣음으로써 $$e_1 * e = e * e_1 = e$$을 얻을 수 있습니다.
      따라서 $e_1 * e = e_1$이고 $e_1 * e = e$임을 알 수 있습니다.
      이것들로부터 $$e = e_1 * e = e_1$$를 얻게 되는데,
      이로써 ($e = e_1$을 보였기 때문에) 증명을 마칠 수 있습니다.
    \end{proof}

    \item 각 원소마다 \underline{역원}(\textit{inverse})\footnote{참고 문헌 \cite{fraleigh2009} 37쪽, 정의 4.1 인용}을 가져야 함:
    각각의 $a \in S$마다, 어떤 $b \in S$가 존재해서
    \begin{equation*}
      a * b = b * a = e \tag{3}
    \end{equation*}
    이어야 합니다.
    이 경우 각각의 $a \in S$에 대하여 (3)을 만족시키는 $b \in S$가 유일하게 존재합니다!
    따라서 그 $b$를 $a^{-1}$과 같이 쓸 수 있고 \underline{$a$의 역원}(\textit{inverse of $a$})라고 부릅니다.

    이를 증명하는 것은 꽤나 어려워 보이지만,
    전과 비슷하게 임의의 역원 후보 $b_1$에 대하여 $b = b_1$일 수 밖에 없음을 보이면 됩니다.

    \begin{proof}
      먼저, $a \in S$를 하나 뽑겠습니다.
      그러면 가정에 의하여 어떤 $b \in S$가 (3)을 만족시킵니다.
      이때 $b_1 \in S$가
      \begin{equation*}
        a * b_1 = b_1 * a = e \tag{3-1}
      \end{equation*}
      를 만족시킨다고 가정한다면,
      \begin{align*}
        b
        & = b * e \tag{by (2)} \\
        & = b * (a * b_1) \tag{by (3-1)} \\
        & = (b * a) * b_1 \tag{by (1)} \\
        & = e * b_1 \tag{by (3)} \\
        & = b_1 \tag{by (2)}
      \end{align*}
      를 얻게 되고,
      증명을 마칠 수 있습니다.
    \end{proof}
  \end{enumerate}

  참고로 결합 법칙을 만족시키는 마그마를 \underline{반군}(\textit{semigroup})이라고 하고,
  항등원이 존재하는 반군을 \underline{모노이드}(\textit{monoid})라고 합니다.\footnote{참고 문헌 \cite{fraleigh2009} 42쪽 인용}
  즉, 군은 각 원소마다 역원을 가지는 모노이드입니다.

  또한 ``\underline{아벨 군}''(\textit{abelian group})의 개념을 소개할까 합니다.
  이항 구조 $\left( S , * \right)$가 주어졌을 때,
  $*$이 \underline{가환}(\textit{commutative})\footnote{참고 문헌 \cite{fraleigh2009} 22쪽, 정의 2.11 인용}하다고 함은
  \begin{equation*}
    \left( \forall a \in S \right) \left( \forall b \in S \right) \left[ a * b = b * a \right] \tag{4}
  \end{equation*}
  이어야 한다는 뜻인데,
  $*$이 가환한 군을 아벨 군이라고 부릅니다.\footnote{참고 문헌 \cite{fraleigh2009} 38쪽, 정의 4.3 인용}

  이제, 군의 예를 하나 공부해 봅시다.

  그 예는 \underline{원군}(\textit{circle group})입니다.
  집합 $S^1$을
  \begin{equation*}
    S^1 := \left\{ z \in \mathbb{C} : \left| z \right| = 1 \right\} \tag{5}
  \end{equation*}
  와 같이 두고,
  복소수의 곱셈으로부터 이항 연산 $* : S^1 \times S^1 \to S^1$을 \underline{유도}(\textit{induce})\footnote{참고 문헌 \cite{fraleigh2009} 21쪽, 정의 2.4 인용}하면,
  $$\left( S^1 , * \right)$$는 군이 됩니다!
  이것이 사실임을 증명할 것인데, 그 전에 다뤄야 할 개념이 있습니다.

  ``$*$은 $S^1$ 위에서 복소수 곱셈의 \underline{유도된 연산}(\textit{induced operation})이다''라고 함은,
  이항 연산 $*$을 복소수의 곱셈의 정의역을 $S^1 \times S^1$으로 줄여서 얻었다는 뜻입니다.
  그런데 이때 $*$이 이항 연산이라는 보장이 있을까요?
  만약 없다면, 우리는 무엇을 확인해 보아야 될까요?

  이항 구조 $\left( S , * \right)$가 주어졌다고 합시다.
  이제 $S$의 부분 집합 $S'$를 하나 잡고, $*'$을
  \begin{align*}
    S' \times S' & \xrightarrow{*'} S' \\
    \left( a , b \right) & \mapsto a * b
  \end{align*}
  로 정의할게요.
  그러면 $*'$이 이항 연산이 되기 위한 조건들 중 (a)와 (c)는, $*$이 이항 연산이라는 가정에 의하여, 이미 성립합니다.
  $*'$는 자신의 정의역에서 $ \left( a , b \right) \mapsto a * b $으로 정의되었으니까요!
  이제 남은 조건은 (b) 뿐인데, 안타깝게도, 이건 성립하지 않을 수도 있습니다.
  따라서 $\left( S' , *' \right)$가 이항 구조임을 보이기 위해서는,
  $S'$에서 (서로 같을 수도 있는) 두 원소 $a$와 $b$를 아무렇게나 뽑아서 $*$에 주더라도,
  그 출력 $a * b$가 $S'$ 밖으로 절대 못 나간다는 것을 보여야 합니다.

  만약 그러하다면, $S'$를 \underline{$*$에 대하여 닫혀 있다}(\textit{closed under $*$})고 합니다.
  즉, 집합 $S' \subseteq S$가 연산 $* : S \times S \to S$에 대하여 닫혀 있을 필요충분조건은
  $$ \left( \forall a \in S' \right) \left( \forall b \in S' \right) \left[ a * b \in S' \right] $$
  가 성립한다는 것입니다.

  이상의 논의로부터 $\left( S^1 , * \right)$가 마그마라는 걸 보이는 것부터 시작해야 되는 걸 알 수 있습니다.
  그렇게 하기 위하여, 두 원소 $z_1 \in S^1$와 $z_2 \in S^1$를 아무렇게나 골라 봅시다.
  그렇더라도, $ z_1 z_2 \in \mathbb{C} $이고 $\left| z_1 \right| = \left| z_2 \right| = 1$이어야 하기 때문에,
  \begin{align*}
    \left| z_1 \right| \left| z_2 \right| = 1
    & \implies \left| z_1 z_2 \right| = 1 \\
    & \implies z_1 z_2 \in S^1
  \end{align*}
  입니다.
  따라서 다음 단계\footnote{$\left( S^1 , * \right)$이 반군임을 증명하기}로 넘어갈 수 있습니다.

  이제 $\left( S^1 , * \right)$이 반군임을 증명해 보겠습니다.
  우리는 이미 복소수의 곱셈이 결합 법칙을 만족시키는 것을 알고 있습니다.
  그런데 $S^1$은 $\mathbb{C}$의 부분 집합입니다.
  그러니까 모든 $a , b , c \in S^1$에 대하여, 
  \begin{equation*}
    (a * b) * c = a * (b * c) \tag{6}
  \end{equation*}
  이 성립하겠죠?
  왜냐하면, (6)을 만족시킨다는 것을 검증받아야 될 모든 삼중쌍 $$ \left( a , b , c \right) \in S^1 \times S^1 \times S^1 $$들은
  전부 집합 $\mathbb{C} \times \mathbb{C} \times \mathbb{C}$의 원소이므로,
  복소수의 곱셈의 결합 법칙이 보증해 주기 때문이죠.
  $*$가 복소수의 곱셈으로부터 유도되었다는 사실을 상기하신다면,
  쉽게 이해가실 거에요.

  이제 $\left( S^1 , * \right)$이 모노이드임을 증명해 보겠습니다.
  $S^1$의 원소들 중 항등원을 하나 찾아서, 그것이 (2)를 만족시킨다는 것을 보이면 충분합니다.
  그런데, 복소수의 곱셈에 대한 항등원인 $1$이 $S^1$에 속해버린 이상, $1$이 $S^1$의 항등원이 되지 않을까요?
  아까와 같은 이유에 의하여, 더 큰 세상에서 보증 받았으므로, 답은 ``예''입니다.

  드디어 $\left( S^1 , * \right)$이 군임을 증명해 보겠습니다.
  그렇기 위해서는, 각 원소 $a \in S^1$마다 어떤 $b \in S^1$가 (3)을 만족시키는 것을 보이면 충분합니다.
  이제 $a \in S^1$를 하나 잡겠습니다.
  그러면 $$ a * \left( 1 / a \right) = \left( 1 / a \right) * a = 1 $$이므로,
  ($1 / a$가 역원임을 보이기 위해서는) $1 / a \in S^1$임만 보이면 충분합니다.
  그런데 $1 / a \in \mathbb{C}$이고 $\left| a \right| = 1$이므로,
  \begin{align*}
    \left| 1 \right| = 1 
    & \implies \left| a \left( 1 / a \right) \right| = 1 \\
    & \implies \left| a \right| \left| 1 / a \right| = 1 \\
    & \implies 1 \left| 1 / a \right| = 1 \\
    & \implies \left| 1 / a \right| = 1 \\
    & \implies 1 / a \in S^1
  \end{align*}
  를 얻게 됩니다.

  이상의 논의로부터 $\left( S^1 , * \right)$를 군이라고 부를 수 있음을 알 수 있습니다.
  추가적으로, 그것은 아벨 군이기도 합니다.
  아까와 마찬가지로, 이 사실도 $*$가 복소수의 곱셈으로부터 유도된 연산임을 이용하면,
  어렵지 않게 보일 수 있습니다.

  노트를 쓰다 보니까 쓸데없는 호기심이 발동하네요.
  모노이드 $\left( M , * \right)$이 주어졌다고 가정하고, $M$의 항등원을 $e$라고 하겠습니다.
  그렇다면 $e \notin M'$인 집합 $M' \subseteq M$을 어떻게 잡든지 $M'$ 안에는 $M'$ 위의 $*$의 유도된 연산에 대한 항등원이 존재하지 않아야 할까요?
  이는 방금 전 $\left( S^1 , * \right)$이 모노이드임을 보일 때와 반대인 상황인데, 답은 ``아니오''입니다.

  \begin{table}[ht]
    \centering
    \label{t1}
    \begin{tabular}{|c|c|c|}
    \noalign{\smallskip}\noalign{\smallskip}\hline
    $*$ & $0$ & $1$ \\
    \hline
    $0$ & $0$ & $1$ \\
    \hline
    $1$ & $1$ & $1$ \\
    \hline
    \end{tabular}
  \end{table}
  반례 중 하나는 다음과 같습니다:
  $M := \left\{ 0 , 1 \right\}$로 두고,
  $*$을 위의 표와 같이 정의하면, 
  $*$에 대한 항등원은 $0$입니다.
  이때 $M' := \left\{ 1 \right\}$로 두면,
  $1 * 1 = 1$이므로 $1$은 $M'$ 위의 $*$의 유도된 연산의 항등원입니다.

  \section{2주차 노트 정리}
  \hspace{12pt}

  앞에서 군의 정의와 군의 예 중 하나인 원군을 배웠습니다.
  이제 다른 예들도 알아봅시다:
  \begin{enumerate}
    
    \item ``정수 덧셈군''이라고 불리는 군 $\left( \mathbb{Z} , + \right)$:

    이 군은 $0$을 항등원으로 가지고,
    각 $i \in \mathbb{Z}$마다 $i \in \mathbb{Z}$의 역원은 $- i$이에요.
    더구나 이 군은 아벨 군입니다.

    \item ``복소수 곱셈군''이라고 불리는 군 $\left( \mathbb{C}^{*} , * \right)$:

    여기서, $\mathbb{C}^{*} := \mathbb{C} \setminus \left\{ 0 \right\}$입니다.
    이 군은 $1$을 항등원으로 가지고,
    각 $z \in \mathbb{C}^{*}$마다 $z^{-1}$을 역원으로 가집니다.
    더구나 이 군은 아벨 군입니다.
    여기서 모든 복소수들의 집합 $\mathbb{C}$에서 $0$을 제외하였는데,
    그 이유는 복소수의 곱셈에 대한 $0$의 역원이라고 부를 수 있는 복소수가 존재하지 않기 때문입니다.
    단, $\mathbb{C}$에서 $0$을 빼냈기 때문에 닫혀 있는지 확인해 봐야 합니다.
    이 군도 정수 덧셈군과 마찬가지로 아벨 군이죠.

    \item 방금까지 원소의 개수가 무한한 군을 살펴 보았는데, 이제 원소의 개수가 유한한 군을 살펴보겠습니다.
    집합 $U_6$를
    \begin{equation*}
      U_6 := \left\{ z \in \mathbb{C} : z^6 = 1 \right\}
    \end{equation*}
    로 두고,
    $*$를 복소수의 곱셈으로부터 $U_6$ 위로 유도된 연산으로 두면,
    $\left( U_6 , * \right)$는 군입니다.
    \begin{figure}[ht]
      \centering
      \begin{tikzpicture}
        \pgfmathsetmacro{\n}{6}
        \begin{axis}[
          axis equal,
          axis lines=center,
          xlabel=$\mathrm{RE}$,
          xtick={-1.5,-0.5,0,0.5,1.5},
          ytick={0,0.5},
          xmax=1.5,
          xmin=-1.5,
          ymax=1.5,
          ymin=-1.5,
          ylabel=$\mathrm{Im}$,
          samples=10,
          disabledatascaling
        ]
          \draw[help lines, black] (0,0) circle (1);
          \foreach \t in {1, ..., \n} {
            \edef\temp{
              \noexpand
              \node[fill=red, circle, draw=red, scale=0.25] at ( {cos((360*\t)/\n)}, {sin((360*\t)/\n)} ) {};
            }\temp
          }
          \foreach \t in {1, ..., \n} {
            \edef\temp{
              \noexpand
              \node[fill=white, circle, draw=none, scale=0.7] at ( {1.25*cos((360*\t)/\n)}, {1.25*sin((360*\t)/\n)} ) {$\zeta^{\t}$};
            }\temp
          }
        \end{axis}
      \end{tikzpicture}
      \caption{복소 평면 위의 $U_6$}
    \end{figure}

    보시다시피 $U_6 = \left\{ \zeta^1 , \zeta^2 , \zeta^3 , \zeta^4 , \zeta^5 , \zeta^6 \right\}$인데,
    여기서 복소수 $\zeta$는 $$\zeta := e^{i \pi / 3} = \frac{1}{2} + \frac{\sqrt{3}}{2} i$$으로 정의되었습니다.
    이때
    \begin{itemize}
      \item 항등원을 $\zeta^6 = 1$으로;
      \item 각 $n \in \left\{ 1 , 2 , \cdots , 6 \right\}$에 대하여
      $ \zeta^n $의 역원을 $ \zeta^{6 - n} = 1 / \zeta^{n} $으로
    \end{itemize}
    두면,
    $\left( U_6 , * \right)$가 군의 정의를 만족시킴을 알 수 있습니다.
    
    \item 그 외에도 $\left( \mathbb{Q} , + \right)$, $\left( \mathbb{R} , + \right)$, $\left( \mathbb{C} , + \right)$은 모두 군이면서 아벨 군이고,

    \item $\left( \mathbb{R}^{*} , * \right)$ 역시 군입니다. (아벨 군이기도 하죠.)
    여기서, $\mathbb{R}^{*} := \mathbb{R} \setminus \left\{ 0 \right\}$입니다.

  \end{enumerate}

  그런데 $\left( \mathbb{R} , + \right)$은 군으로서 $\left( \mathbb{C} , + \right)$ 안에 들어갈 수 있네요?
  ``군으로서 안에 들어갈 수 있다''라는 개념을 정의하려면, ``부분군''과 ``군 동형''의 개념이 필요합니다.

  먼저 \underline{부분군}(\textit{subgroup})\footnote{참고 문헌 \cite{fraleigh2009} 49쪽, 정의 5.4 인용}에 대하여 알아 보겠습니다.
  만약 군 $G$의 부분 집합 $H$가 $G$의 이항 연산 아래에 닫혀 있고,
  $G$로부터 유도된 연산을 가지는 $H$가 그 자신이 군이면 $H$는 $G$의 부분군이라고 합니다.
  $H$가 $G$의 부분군인 경우, 그 사실을 ``$H \leq G$'' 또는 ``$G \geq H$''라고 적어서 나타냅니다.

  군 $G$의 항등원 $e$이라면, 당연히 집합 $\left\{ e \right\}$는 $G$의 부분군입니다.
  (심지어 아벨 군이기도 합니다.)
  이를 $G$의 \underline{자명한}(\textit{trivial}) 부분군이라고 합니다.\footnote{참고 문헌 \cite{fraleigh2009} 49쪽, 정의 5.5 인용}
  자명한 부분군이 아닌 부분군을 가르켜 \underline{비자명}(\textit{nontrivial}) 부분군이라고 부릅니다.\footnote{참고 문헌 \cite{fraleigh2009} 49쪽, 정의 5.5 인용}

  $G$가 군이면 $G$ 자신도 $G$의 부분군입니다.
  $G$를 $G$의 \underline{비진부분군}(\textit{improper subgroup})이라고 합니다.\footnote{참고 문헌 \cite{fraleigh2009} 49쪽, 정의 5.5 인용}
  $G$가 아닌 $G$의 부분군을 가르켜 \underline{진부분군}(\textit{proper subgroup})이라고 부릅니다.\footnote{참고 문헌 \cite{fraleigh2009} 49쪽, 정의 5.5 인용}

  그런데 언제 부분 집합이 부분군이 되는지 생각해 봅시다.
  그러면 어떠한 군 $ \left( G , * \right) $가 주어지더라도, 임의의 집합 $H \subseteq G$가 $G$의 부분군일 때 그리고 그럴 때에만 다음 조건들이 모두 성립함을 알 수 있습니다:\footnote{참고 문헌 \cite{fraleigh2009} 51쪽, 정의 5.14 인용}
  \begin{enumerate}
    \item \underline{$H$가 $G$의 이항 연산 아래에 닫혀 있다.}
    다시 말해, 논리식
    \begin{equation}
      \left( \forall a \in H \right) \left( \forall b \in H \right) \left[ a * b \in H \right] \tag{7}
    \end{equation}
    가 성립한다는 뜻입니다.
    \item \underline{$G$의 항등원이 $H$에 속한다.}
    다시 말해, 논리식
    \begin{equation}
      e \in H \tag{8}
    \end{equation}
    가 성립한다는 뜻입니다.
    \item \underline{각 $a \in H$에 대하여, $a$의 역원이 $H$에 속한다.}
    다시 말해, 논리식
    \begin{equation}
      \left( \forall a \in H \right) \left( \exists b \in H \right) \left[ a * b = b * a = e \right] \tag{9}
    \end{equation}
    가 성립한다는 뜻입니다.
  \end{enumerate}

  \begin{proof}
    먼저, 파트 (a)에서 필요한 보조 정리를 하나 증명하고;
    그 다음, 파트 (b)에서 $H \leq G$이 성립한다고 가정한 뒤 조건 1, 2, 3이 성립함을 보이고;
    마지막으로, 파트 (c)에서 조건 1, 2, 3이 성립한다고 가정한 뒤 $H \leq G$임을 보이겠습니다.

    \begin{itemize}
      \item[(a)] 군 $\left( G , * \right)$가 주어졌다고 하겠습니다.
      그러면 임의의 $a \in G$와 임의의 $b \in G$에 대하여 (10)과 (11)이 성립합니다:\footnote{참고 문헌 \cite{fraleigh2009} 40쪽, 정의 4.15 인용}
      \begin{align*}
        \left( \exists x \in G \right) \left[ a * x = b \land \left( \forall y \in G \right) \left[ a * y = b \rightarrow x = y \right] \right] \tag{10} \\
        \left( \exists x \in G \right) \left[ x * a = b \land \left( \forall y \in G \right) \left[ y * a = b \rightarrow x = y \right] \right] \tag{11}
      \end{align*}
      다시 말해, $x$에 대한 방정식 $a * x = b$은 유일한 해를 가지고,
      또 다른 $x$에 대한 방정식 $x * a = b$도 유일한 해를 가진다는 뜻입니다.

      \begin{proof}[Proof of (a)]
        먼저, (10)이 성립하는 것부터 보이겠습니다.
        $x := a^{-1} * b$로 두면,
        \begin{align*}
          a * x
          & = a * \left( a^{-1} * b \right) \\
          & = \left( a * a^{-1} \right) * b \\
          & = e * b \\
          & = b
        \end{align*}
        이고,
        $a * y = b$인 임의의 $y \in G$에 대하여,
        \begin{align*}
          a * x = b
          & \implies a * x = a * y \\
          & \implies a^{-1} * \left( a * x \right) = a^{-1} * \left( a * y \right) \\
          & \implies \left( a^{-1} * a \right) * x = \left( a^{-1} * a \right) * y \\
          & \implies e * x = e * y \\
          & \implies x = y
        \end{align*}
        이므로, (10)이 성립함을 알 수 있습니다.

        이제, (11)이 성립함을 보이겠습니다.
        아까와 비슷하게 $x := b * a^{-1}$로 두면,
        \begin{align*}
          x * a
          & = \left( b * a^{-1} \right) * a \\
          & = b * \left( a^{-1} * a \right) \\
          & = b * e \\
          & = b
        \end{align*}
        이고,
        $y * a = b$인 임의의 $y \in G$에 대하여,
        \begin{align*}
          x * a = b
          & \implies x * a = y * a \\
          & \implies \left( x * a \right) * a^{-1} = \left( y * a \right) * a^{-1} \\
          & \implies x * \left( a * a^{-1} \right) = y * \left( a * a^{-1} \right) \\
          & \implies x * e = y * e \\
          & \implies x = y
        \end{align*}
        이므로, (11)이 성립함을 알 수 있습니다.
      \end{proof}

      \item[(b)] 먼저 $H \leq G$라고 가정합시다.
      그러면 $H$는 $G$의 부분군이므로, 부분군의 정의에 의하여,
      $*$에 대하여 닫혀 있고;
      $H$ 위에서 $*$의 유도된 이항 연산 $*'$에 대하여, $\left( H , *' \right)$는 그 자체로 군이어야 합니다.

      여기서 $H$가 $*$에 대하여 닫혀 있으므로 조건 1이 성립함을 알 수 있습니다.
      
      또한 $H$는 군이기 때문에,
      공집합이 아니어야 하므로 어떤 $h_0 \in H$가 존재함을 알 수 있고;
      보조 정리 (a)를 적용하여 어떤 $e' \in H$가 존재하여 $$ h_0 *' e' = h_0 \land \left( \forall h \in H \right) \left[ h_0 *' h = h_0 \rightarrow e' = h \right] $$가 성립함을 알 수 있습니다.
      이때 보조 정리 (a)를 군 $ \left( G , * \right) $에 적용하면, 어떤 $x \in G$가 $$ h_0 * x = h_0 \land \left( \forall y \in G \right) \left[ h_0 * y = h_0 \rightarrow x = y \right] $$를 만족시킴을 알 수 있는데,
      $y$에 $G$의 항등원 $e$를 넣음으로써 $x = e$를 얻을 수 있습니다.
      이제 $y$에 $e'$를 넣으면 $x = e'$가 도출되는데, $e = e' \in H$를 얻게 됩니다.
      따라서 조건 2가 성립함을 알 수 있습니다.
      물론, $e$가 $*'$에 대한 항등원임도 쉽게 증명할 수 있습니다.
      귀뜸만 하자면, 더 큰 군에서도 항등원으로서의 의무를 수행하며 살았는데 걔가 부분군으로 이사 가더라도 여전히 (오히려 더 가벼운) 항등원으로서의 의무를 수행할 수 있다는 말을 증명으로 잘 바꾸면 된다고 말씀드리겠습니다.

      조건 3이 성립함을 보이기에 앞서 주장 [1]을 증명하겠습니다.
      \begin{enumerate}
        \item[{주장 [1]:}] $x * y = e$를 만족시키는 임의의 $x \in G$와 $y \in G$에 대하여 $y * x = e$이다.
      \end{enumerate}

      \begin{proof}[Proof of the claim]
        $x \in G$와 $y \in G$가 $x * y = e$를 만족시킨다고 가정하겠습니다.
        이때 $z := y * x$로 두고, $z * z = z$를 보이면 충분합니다.
        그 까닭은 양변에 $z^{-1}$을 곱하여 $z = e$를 얻을 수 있기 때문입니다.
        그런데
        \begin{align*}
          z * z
          & = \left( y * x \right) * \left( y * x \right) \\
          & = y * \left( x * \left( y * x \right) \right) \\
          & = y * \left( \left( x * y \right) * x \right) \\
          & = y * \left( e * x \right) \\
          & = y * x \\
          & = z
        \end{align*}
        가 도출되네요.
        이로써 주장 [1]의 증명이 끝났습니다.
      \end{proof}
      
      이제 $H$에서 원소 하나를 뽑아 $a$라고 부르겠습니다.
      그러면 보조 정리 (a)에 의하여 어떤 $b \in H$가 존재하여 $$ a *' b = e \land \left( \forall h \in H \right) \left[ a *' h = e \rightarrow b = h \right]$$가 성립합니다.
      이제 보조 정리 (a)를 군 $ \left( G , * \right) $에 적용하면, 어떤 $x \in G$가 $$ a * x = e \land \left( \forall y \in G \right) \left[ a * y = e \rightarrow x = y \right] $$를 만족시킴을 알 수 있는데,
      $y$에 $b$를 넣음으로써 $x = b$를 얻습니다.
      따라서 $a * b = e$를 얻게 되고, 주장 [1]에 의하여 $b * a = e$를 얻습니다.
      이로써 조건 3의 증명이 끝났습니다.

      \item[(c)] 이제 $G$의 부분 집합인 $H$가 조건 1, 2, 3을 모두 만족시킨다고 가정하겠습니다.
      
      먼저, $H$가 $*$ 아래에 닫혀 있음을 보여야 하는데, 이는 조건 1이 보장합니다.

      이제 $*'$을 $H$ 위에서 $*$의 유도된 이항 연산이라고 하겠습니다.
      그러면 $\left( H , *' \right)$는 이항 구조가 되는데, 이것이 군임을 보이면 충분합니다.
      그런데 이는 $S^1$이 군임을 보였던 과정을 반복하여 증명할 수 있습니다.
    \end{itemize}
  \end{proof}

  여기서 흥미로운 점들 중 하나는 (거칠게 말하면) 보조 정리 (a)의 역이 성립한다는 사실입니다.
  정확히 말씀드리자면, 조건 (10)과 (11)을 만족시키는 임의의 반군 $ \left( G , * \right) $은 군입니다.
  이를 증명하기 앞서 더 재밌는 정리를 증명해 보겠습니다!

  \bibliographystyle{unsrt}

  \bibliography{Abstract_Algebra_1_and_Laboratory}

\end{document}
